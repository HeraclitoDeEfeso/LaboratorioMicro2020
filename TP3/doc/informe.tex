\documentclass[a4paper]{article}
% Eso define el tipo de documento.
% `article` se divide en secciones y tiene el resumen pegado al texto
\usepackage{graphicx}
% Son como los import de librerias. `graphicx` es para importar imagenes
\usepackage{listings}
% `listings` es para insertar los codigo fuente
\usepackage[utf8]{inputenc}
% `utf8` es para no tener problemas con el encoding (acentos, etc.)
\usepackage{caption}
% Para cambiar leyenda de figuras

\title{Trabajo Práctico Nro. 3:\\``Sistema de llenado\\y vaciado de un tanque''}
\author{Amodey, Leandro - leandroamodey@gmail.com
\and Monti, Matías - matiasmonti@hotmail.com
\and Quinteros, Fernando - lordfers@gmail.com
\and Araneda, Alejandro – eloscurodeefeso@hotmail.com}
\date{1er. Cuatrimestre 2020\\Jueves, 25 de Junio}
% `title`, `author` y `date` es informacion que todos los archivos tienen.

\def\teacher{Ing. Jorge H. Doorn
\and Ing. Matías Presso}
% Variable propia para definir profesores.

\captionsetup{justification=centering,labelsep=period,font={small},%
labelfont=bf,textfont=it}
% Las leyendas deben ser chicas y en italica, separadas con punto
% y la numeración en negritas, todo centrado.
\renewcommand{\figurename}{Figura}
% Con eso cambiamos como salen los titulos de las figuras.
\renewcommand{\abstractname}{Resumen}
% Con eso le cambio en titulo del resumen
\renewcommand{\refname}{Referencias}
% Cambia el titulo de la bibliografia
\let\originalcite\cite
\renewcommand{\cite}[2][]{\textsuperscript{\originalcite{#2}}}
% Cambiamos el estilo de las citas bibliograficas
\makeatletter\let\@afterindentfalse\@afterindenttrue\makeatother
% Cambiamos \@afterindentfalse por \@afterindenttrue para indentar primer parrafo
\let\originalappendix\appendix
\renewcommand{\appendix}{%
    \newpage\originalappendix\pagenumbering{gobble}%
    % Empezar apéndices en nueva pagina sin numeracion
    \renewcommand\thesection{Anexo \Alph{section}}
    % `thesection` es como se numeran los apendices
    \setcounter{secnumdepth}{1}
    % Contarlas secciones de apendice
}
\setcounter{secnumdepth}{0}
% No se numeran las secciones pero si estan en el TOC 
\newenvironment{ejercicios}
    {\setcounter{secnumdepth}{3}
    \renewcommand\thesubsection{Ejercicio \arabic{subsection}}}
    {\setcounter{secnumdepth}{0}}
% Defino un enviroment para numerar ejercicios

\begin{document}
% Arranca el documento

% Pagina propia de titulo
\begin{titlepage}\renewcommand\and\par\centering\makeatletter
%    \includegraphics{logo.png}\par
    {\Large Ingeniería en Computación \par}\vspace{0.5cm}
    {\LARGE Laboratorio de Microprocesadores \par}\vfill
    {\huge \@title \par}\vfill
    Grupo 2:\par
    \@author\vfill
    Práctica entregada:\par
    \@date\vfill
    Docentes:\par
    \teacher\vspace{1cm}\makeatother
\end{titlepage}

\begin{abstract}

    En el presente trabajo realizaremos el análisis y diseño para la implementacion
    de un sistema que se ocupe del llenado y vaciado de un tanque mediante el uso de 
    microcontroladores y elementos eléctrìcos y electrónicos.
    Como entregable del presente trabajo se realizará el presente informe junto con los archivos 
    del esquemático del sistema, y una simulación del mismo funcionando.

\end{abstract}

\section{Introducción}


\subsection*{Características de los Puertos Analógicos}

En la Figura \ref{fig:analog-pin} se muestra un circuito simplificado
para una entrada analó-gica. Como los pines analógicos están conectados
a una salida digital, tienen diodos con polarización inversa a VDD y 
VSS. La entrada analógica, por lo tanto, debe estar entre VSS y VDD.

\begin{figure}[h]\centering
%    \includegraphics[height=3cm]{informefig4.png}
    \caption{Ilustración del circuito interno de un puerto analógico}
    \label{fig:analog-pin}
\end{figure}

Si el voltaje de entrada se desvía de este rango en más de 0.6V
en cualquier dirección, uno de los diodos está polarizado hacia adelante
y puede producirse un bloqueo. Se recomienda una resistencia máxima de 10 kOhm para las fuentes analógicas.
Cualquier componente externo conectado a un pin de entrada analógica,
como un capacitor o un diodo Zener, debe tener muy poca pérdida de corriente.

\subsubsection*{Módulo Comparador Análógico}

El PIC12F675 tiene un comparador analógico, y las entradas al
comparador se multiplexan con los pines GP0 y GP1.
Además, GP2 se puede configurar como salida del comparador.
El registro de control del comparador (CMCON) contiene
los bits para controlarlo.
		
\begin{figure}[h]\centering
%    \includegraphics[height=4cm]{informefig3.png}
    \caption{Resultado del comparador según los niveles de señales}
    \label{fig:analog-signal}
\end{figure}

En la Figura \ref{fig:analog-signal} se muestra el resultado de la 
comparación según el nivel de la señales. Cuando la entrada 
analógica en VIN+ es menor a la entrada
analógica en VIN-, generará una salida de nivel bajo, pero si VIN+
es mayor que VIN-, generará un nivel alto.
Para poder trabajar con el comparador
utilizando la entrada analógica AN0 y el voltaje de referencia VREF 
en lugar de GP0 y GP1, hay que setear los bits en el registro CMCON 
en {\ttfamily 00011101} o {\ttfamily 1D} en hexadecimal.

\subsubsection*{Módulo Conversor Analógico/Digital}

Este módulo maneja un formato de conversión de 10 bits. Para el 
control, hay 2 registros disponibles para controlar las 
funcionalidades del módulo de conversión analog-to-digital. Además, 
este módulo se maneja a través de los "bit-comparator" analógicos
ANSEL (de 0 a 3) y los bits del TRISIO que controlan el 
funcionamiento de los pines del mismo (normalmente se utiliza para 
configurarlo en OUTPUT), y análogamente seteamos el bit ANS 
correspondiente para deshabilitar el INPUT buffer.

\subsection*{Dispositivos con corriente considerable o corriente alterna}

La corriente alterna por su alto voltaje puede ser peligrosa y requiere un 
aislamiento muy superior.No se puede almacenar y puede producir pulsos 
electromagnéticos que afecten a equipos electrónicos sensibles como un radio
 o celular.

Para evitar que una corriente de alto voltaje dañe los dispositivos podemos
usar transistores que dependiendo la situacion pueden funcionar como amplificador
de intensidad o como regulador ya que no permitira que pase mas de cierta cantidad
corriente.

El transistor cuenta con 3 pines el emisor se encarga de proporcionar las cargas 
eléctricas, la base controla el flujo de corriente y por ultimo el colector recoge
las cargas proporcionadas por el emisor. La diferencia de usos entre transistores 
es que los NPN se utilizan para voltajes positivos y los PNP con voltajes negativos

\begin{figure}[h]\centering
%    \includegraphics[height=3cm]{transistorilustrado.png}
    \caption{Representacion de un trasistor}\label{fig:transistorilustrado}
\end{figure}


Estos dispositivos tienen 3 estados:

\begin{itemize}
    \item {
	Se dice que un transistor entra en estado de corte cuando el voltaje
	de la base es nulo o menor a 0.6v, ya que que no logra activar el paso
	de corriente entre el colector y el emisor, es decir se comporta como 
	un interruptor abierto.
    }
    \item {
        El funcionamiento en el estado de saturacion es el caso contrario a la 
	de corte, ya que cuando el voltaje que circula por la base supera al 
	establecido por el fabricante, satura al transistor y este permite la
	 circulación entre colector y emisor como si fuera un cable normal,
	 es decir se comporta como un interruptor cerrado.
    }
    \item {
        El estado activo se logra cuando el voltaje de la base esta en un rango
	intermedio entre la región de saturación y la de corte. 
	Cuando logramos estabilizar el transistor es capaz de amplificar las 
	señales de entrada las veces que tenga el valor de $\beta$ ya que este 
	multiplica la corriente del transistor.
    }    
\end{itemize}

\subsection*{Tecnologías CMOS y TTL}
% Fijarse que este tiene * para que no se numere.

Por tecnología TTL (en inglés, \textit{transistor-transistor logic})
nos referimos a aquella que utiliza transistores de unión bipolar o 
BJT (en inglés, \textit{bipolar junction transistor}), para la 
construcción de circuitos digitales\cite{bib:boylestad}.

En cambio, los CMOS (en inglés, \textit{complementary 
metal-oxide-semiconduc-tor}) son un tipo de implementación de 
transistores de efecto de campo de la familia de los MOSFET (en 
inglés, \textit{metal-oxide-semiconductor field-effect transistor}). 
La denominación se ha mantenido a pesar de haberse generalizado el 
uso de silicio en vez de metal y otros aislantes en reemplazo del 
óxido.

\begin{figure}[h]\centering
%    \includegraphics[height=3cm]{transistores.png}
    \caption{Integrados con tecnologías CMOS (izquierda) y TTL 
    (derecha)}\label{fig:transistores}
\end{figure}

\subsubsection*{Principales Diferencias}

En la Figura \ref{fig:transistores} mostramos ejemplos de integrados 
para ambas tecnologías y a continuación listaremos sus principales 
diferencias desde la perspectiva de nuestras aplicaciones a diseñar:

\begin{itemize}
    \item {
        Los integrados desarrollados con CMOS son en general más 
        costosos que su contraparte TTL.
    }
    \item {
        Los integrados con tecnología TTL son más robustos contra
        las descargas de estáticas.
    }
    \item {
        La propagación es más rápida en los TTL que en los CMOS. 
        Estas últimas características hacen que los TTL sigan siendo
        utilizados.
    }    
    \item {
        La dimensión de los CMOS es mucho menor que la de los TTL.
    }
    \item {
        Al operar con baja corriente, los CMOS tienen menos consumo 
        que los TTL. Estas últimas características han hecho que los 
        CMOS se transformaran en los más utilizados en la industria.
    }
    \item {
        Además, los niveles de tensión para los integrados con TTL 
        rondan los 5 voltios mientras los márgenes para los CMOS son
        mucho mayores: desde los 3 a los 18 voltios.
    }
    \item {
        La cantidad de salidas que pueden conectarse a un CMOS sin 
        alterar el correcto funcionamiento (puesto que funcionan con
        baja corriente) es cinco veces superior a las que permite un 
        TTL. Sin embargo, la cantidad de entradas posibles son 
        levemente superiores en estos últimos.
    }

\end{itemize}


\section{Descripción de la Práctica}

\subsection{Enunciado}

A continuación transcribimos el enunciado original de la práctica.
Del mismo tomamos los puntos teóricos que son descriptos en la 
Introducción. Los ejercicios prácticos son desarrollados en la sección
Diseño y Simulación.

\begin{quotation}

    \begin{center}

        Laboratorio de Microprocesadores - 2020
        
        Taller de Microprocesadores Trabajo Practico  2                

    \end{center}

    \begin{enumerate}

        \item{
            Describir brevemente las características de las entradas 
            analógi-cas del chip. Determine cuantos niveles se pueden 
            detectar en una entrada analógica y cual es la mínima 
            variación que se puede detectar.
            
            Haga uso de una entrada analógica en el entorno de Proteus,
            detecte un nivel de amplitud de una señal analógica 
            conectada a una de las entradas del chip. Puede detectar 
            si es superior o inferior a cierto valor.
            
            Por ejemplo: puede encender un LED cuando el nivel de una 
            señal analógica supere cierto valor.

            Experimente el uso de instrumentos en el entorno de Proteus.

            Los instrumentos en la simulación serán de mucha utilidad 
            para verificar el funcionamiento del diseño y del programa. 
            Encontrará, entre otros, voltímetro, amperímetro, 
            voltímetro, generador de ondas analógicas, osciloscopio, 
            etc.
        }
        \item{
            Describa brevemente qué es la tecnología CMOS, qué es la 
            tecnología TTL y qué las diferencia.
        }
        \item{
            Qué dispostivo/s emplearía para controlar con el 
            microcontrolador el encendido y apagado de un dispositivo 
            que funciona a una corriente considerable o que funciona 
            con corriente alterna. Implemente un circuito en Proteus. 
            (Puede modificar el circuito realizado para el item 3 del 
            TP1, para encender una lámpara o un motor).
        }
        \item{
            Diseñe el circuito de alimentación del chip PIC12F675 
            considerando que tendrá una fuente de alimentación externa 
            de 9V. La fuente externa es una fuente 220 Vac / 9 Vdc.
        } 
    \end{enumerate}
\end{quotation}

\subsection{Plataforma de Desarrollo}

Utilizamos el lenguaje C para programar las aplicaciones. Presentamos
una copia de los mismos como anexos.

Para nuestro desarrollo utilizamos el compilador MPLAB 
XC8\cite{bib:compilador} de la empresa Microchips. El mismo es el 
diseñado específicamente para la línea de microcontroladores de 8 bits
a la que pertenece el PIC12F675.

El diseño y simulación del esquemático correspondiente a cada 
aplicación se realiza con el software Proteus\cite{bib:simulador} de 
la compañía Labcenter Electronics.

\subsection{Instrucciones de Compilación y Ejecución}

Para la compilación del firmware utilizamos la línea de comandos en 
una terminal. Como parámetro a la ejecución del compilador agregamos
el modelo del microcontrolador donde se instala el software. Éste es 
el método recomendado por el desarrollador por sobre el de incluir en 
el código mismo los archivos de encabezado para el preprocesador con 
las configuraciones específicas del modelo. Por ejemplo, para el 
primer ejercicio utilizamos:

\begin{center}\ttfamily 
	xc8 --chip=12f675 ejercicio1.c
\end{center}

El compilador genera un archivo {\ttfamily .hex} que es el que
agregamos a las propieda-des del microcontrolador solicitadas 
por el software Proteus para la simulación del circuito. Allí también 
indicamos tanto la frecuencia del reloj y la palabra que representa 
los bits de configuración que debieran ser impresos en la memoria del
integrado junto con el código ejecutable. 

\section{Diseño del sistema}

\begin{ejercicios}

    \subsection{Consideraciones generales}\label{ej:monitoreo}

    Completar

    % --- Para insertar una imagen escriban de acá
    \begin{figure}[h]\centering
    % La opcion `[h]` es para "here", pero puede cagarse en eso.
    % No usar "la imagen que sigue" sino la referencia.
        \includegraphics[height=10cm]{diagrama_sistema.jpg}
        \caption{Esquema del sistema general.}\label{fig:esquematico1}
    \end{figure}
    % --- hasta acá.

    
\end{ejercicios}


\section{Conclusiones}

En general, los microcontroladores están destinados a aplicaciones
específicas que siempre involucran un intercambio con un medio que 
no es digital. El manejo tanto de señales analógica como de 
alimentacion de corriente alterna resulta indispensable.

Los transistores son los componentes indispensables para esta 
interconexión, lo mismo que diversas construcciones como las 
escaleras de resistencias para la conversion analógica/digital
o las tecnicas de administración de energía como la modulación
por pulso (o PWM, por sus siglas en inglés).

Esta práctica brinda un panorama rápido acerca de los componentes
electró-nicos involucrados, así como la configuración necesaria de 
los registros y módulos del microcontrolador que han sido destinados
específicamente para estas aplicaciones.

\noindent\rule{\textwidth}{1pt}
% Linea horizontal sin identacion y del ancho del texto

\begin{thebibliography}{9}
% Comienza la bibliografia
% El "9" indica que el espacio para imprimir "9" es de la mayor referencia

% Para agregar una entrada a la bibliografia repetir de aca
\bibitem{bib:boylestad} 
Boylestad, R. \& Nashelsky, L. (2002). 
\textit{``Electronic devices and circuit theory"}.
Upper Saddle River, N.J: Prentice Hall.
% Hasta aca termina la referencia

\bibitem{bib:compilador}
\textit{``Microchip MPLAB XC8 C Compiler"}
(Versión 2.10; Microchip Technology Inc.: 2019).
Recuperado de https://www.microchip.com/mplab/compilers

\bibitem{bib:simulador}
\textit{``Proteus 8 Professional"} 
(Versión 8.5 Service Pack 0; Labcenter Electronics: 2016).
Recuperado de https://www.labcenter.com/

\bibitem{bid:datasheet}
Microchip (2010).
\textit{``PIC12F629/675 Data Sheet. 8-Pin FLASH-Based 8-Bit CMOS 
Microcontrollers''.}
EE.UU. Recuperado de 
http://ww1.microchip.com/downloads/en/DeviceDoc/41190G.pdf

\bibitem{bid:Transistor}
``¿Qué es un transistor y como funciona?'' (2020).
Recuperado de 
https://www.ingmecafenix.com/electronica/el-transistor/

\end{thebibliography}


\appendix

\section{}\label{ane:monitoreo}
A continuación listamos en extenso el código fuente en lenguaje C
correspondiente al \ref{ej:monitoreo}. La descripción de su 
funcionamiento puede encontrarse en la sección de Diseño y 
Simulación.

%\lstinputlisting[language=C,basicstyle=\ttfamily\scriptsize,numbers=left]{../ejercicio1.c}
% `lstinputlisting` inserta el codigo.
% Lo configuro para C, en monospace y tamaño chico, con numeracion a la izquierda.
% Recordar que el archivo tiene una direccion relativa a esta carpeta.
% Los de verdad van a estar en la carpeta padre "../ejercicio1.c"

\newpage
% Incluir salto de pagina de manera manual en cada apendice.
\section{}\label{ane:Corriente alterna o considerable}
A continuación podran ver el  código fuente en lenguaje C correspondiente
al \ref{ej:Resistor}. La descripción de su funcionamiento puede encontrarse en
la sección de Diseño y Simulación.

%\lstinputlisting[language=C,basicstyle=\ttfamily\scriptsize,numbers=left]{../ejercicio2.c}

\end{document}
