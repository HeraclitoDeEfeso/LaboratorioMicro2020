\documentclass[a4paper]{article}
\usepackage{graphicx}
\usepackage{listings}
\usepackage[utf8]{inputenc}
\usepackage{caption}
\usepackage{multirow}
\usepackage{pdflscape}
\usepackage{afterpage}

\title{Trabajo Práctico Nro. 3:\\``Sistema de llenado\\y vaciado de un tanque''}
\author{Amodey, Leandro - leandroamodey@gmail.com
\and Monti, Matías - matiasmonti@hotmail.com
\and Quinteros, Fernando - lordfers@gmail.com
\and Araneda, Alejandro – eloscurodeefeso@hotmail.com}
\date{1er. Cuatrimestre 2020\\Jueves, 26 de Junio, 11hs.}

\def\teacher{Ing. Jorge H. Doorn
\and Ing. Matías Presso}

\captionsetup{justification=centering,labelsep=period,font={small},%
labelfont=bf,textfont=it}
\renewcommand{\figurename}{Figura}
\renewcommand{\abstractname}{Resumen}
\renewcommand{\refname}{Referencias}
\let\originalcite\cite
\renewcommand{\cite}[2][]{\textsuperscript{\originalcite{#2}}}
\makeatletter\let\@afterindentfalse\@afterindenttrue\makeatother
\let\originalappendix\appendix
\renewcommand{\appendix}{%
    \newpage\originalappendix\pagenumbering{gobble}%
    \renewcommand\thesection{Anexo \Alph{section}}
    \setcounter{secnumdepth}{1}
}
\setcounter{secnumdepth}{0}

\begin{document}

\begin{titlepage}\renewcommand\and\par\centering\makeatletter
    \includegraphics{logo.png}\par
    {\Large Ingeniería en Computación \par}\vspace{0.5cm}
    {\LARGE Laboratorio de Microprocesadores \par}\vfill
    {\huge \@title \par}\vfill
    Grupo 2:\par
    \@author\vfill
    Práctica entregada:\par
    \@date\vfill
    Docentes:\par
    \teacher\vspace{1cm}\makeatother
\end{titlepage}

\begin{abstract}

    En el presente trabajo realizamos el análisis y diseño de la
    implementación de un sistema que se ocupa del llenado y vaciado 
    de un tanque mediante el uso de microcontroladores y otros 
    componentes electrónicos.

\end{abstract}

\section{Introducción}

Para este trabajo utilizamos los conocimientos adquiridos en las 
prácticas anteriores y sumamos nuevas herramientas investigadas, 
que a continuación describimos.

% \subsubsection*{Sensor}

% En primer lugar como la consigna aclara que tanto para vaciar y llenar 
% el tanque este debe estar entre ciertos niveles de agua debemos contar 
% con una herramienta que nos permita saber la cantidad actual del tanque,
%  para esto utilizaremos un sensor del nivel de agua.

% Este sensor se alimenta a 5V o a 3,3V, el pin S nos dará un valor 
% analógico de manera que usaremos el pin S conectándolo como entrada
%  analógica, el valor que leamos será mayor en función de la superficie 
% del sensor este cubierto de agua. Esto se debe a que el agua se comporta 
% como un conductor, teniendo en cuenta que el agua que usemos en nuestros 
% depósitos no será agua pura (H2O), ya que de por si el agua no es 
% conductora.

\subsubsection*{Conversor digital analógico}

Un conversor o convertidor digital analógico (\textit{DAC}, por sus 
siglas en inglés) del tipo ``R-2R'' como el que se muestra el la 
Figura \ref{fig:conversor}, suma varias señales digitales binarias 
de acuerdo al peso de cada una dando como resultado una señal de 
corriente o tensión analógica.

\begin{figure}[h]\centering
    \includegraphics[height=4cm]{conversor.png}
    \caption{Convertidor digital analógico (DAC) R-2R}
    \label{fig:conversor}
\end{figure}

Se denominan también ``escaleras de resistencias'' por la forma que 
tiene el circuito. Gracias a esta tipología, ingresa al 
microcontrolador un valor analógico que luego será utilizado para hacer 
comparaciones en la lógica del programa. Presenta algunas 
ventajas como la alta velocidad de conversión o el funcionamiento 
sencillo, mientras que por otro lado exige que los valores de las 
resistencias sean precisos, sobre todo los de las resistencias de los
bits MSB (\textit{Most Significant Bit}, en inglés) además de que 
son necesarias muchas resistencias. El ``R-2R'' brinda un voltaje a 
la salida que va entre 0 a 5 V si se considerada que las entradas 
digitales pueden brindar únicamente esos dos valores y tendrán $2^n$ 
valores analógicos posibles.

% \subsubsection*{Bombas de Agua}

% Una bomba de agua es una máquina hidráulica que permite incrementar 
% la energía cinética de un caudal de agua. Las bombas hidráulicas son 
% elementos ampliamente conocidos y empleados
%  en la industria desde antaño, y constituyen toda una rama de la 
%  técnica. Existe una gran variedad de bombas, que abarcan un amplio 
%  rango de potencias y características hidráulicas.

% Independientemente de sus características o potencia, siempre podemos 
% controlar un equipo de bombeo mediante un procesador, siendo de hecho 
% frecuente que estén controlados por un autómata. Arduino, por 
% supuesto, no es una excepción, y podemos encender cualquier tipo de 
% bomba de agua mediante las salidas digitales y el uso de un MOSFET o 
% una salida por relé.

\subsubsection*{Relé}

Un relé es un interruptor accionado eléctricamente. Muchos relé 
utilizan un electroimán para operar mecánicamente un interruptor, 
pero otros principios de funcionamiento también se utilizan los 
relés de estado sólido. Los relés se utilizan cuando es necesario 
controlar un circuito con una señal de baja potencia (obteniendo
un aislamiento eléctrico completo entre el control y los circuitos 
controlados), o cuando varios circuitos deben ser controlados por 
una sola señal.

\begin{figure}[h]\centering
    \includegraphics[height=4cm]{rele.png}
    \caption{Símbolo para un relé en un esquemático}
    \label{fig:Rele}
\end{figure}

En la Figura \ref{fig:Rele} se observa un esquema del relé. Su 
funcionamiento es el siguiente: al ingresar corriente por la bobina 
los contactos abiertos se cierran y los cerrados se abren.

\subsubsection*{Optoacoplador}

El propósito principal de un optoacoplador es proteger al circuito 
de salida frente a picos de voltajes o tensiones elevadas en su 
entrada que pueden dañar al otro circuito. Un optoacoplador 
contiene, por lo general, un LED que convierte la señal eléctrica 
de entrada en luz y un sensor que detecta la luz del LED.

\begin{figure}[h]\centering
    \includegraphics[height=4cm]{Optoacoplador.png}
    \caption{Circuito equivalente al optoacoplador.}
    \label{fig:Optoacoplador}
\end{figure}

En la Figura \ref{fig:Optoacoplador} se muestra como funciona el 
Optoacoplador en sus dos estados. Como se observa, el mismo 
cuenta con un emisor y un receptor. El emisor es 
un led infrarrojo que manda un haz de luz hacia el receptor que 
normalmente es un fototransistor, cuando este dispositivo capta 
la señal actúa como un interruptor cerrado y cuando se interrumpe 
actúa como un interruptor abierto.

\section{Descripción de la Práctica}

\afterpage{
\begin{landscape}
    \begin{table}[p]\centering
        \begin{tabular}{|c|c|c|c|c|cccc|c|}
            \hline
            \multirow{2}{*}{Descripción}      & \multirow{2}{*}{Marca}      & \multirow{2}{*}{Modelo} & \multicolumn{2}{c|}{Alimentación / Control}                                                                                 & \multicolumn{4}{c|}{Respuesta}                                                                                                                                                                                                                                                                                                & \multirow{2}{*}{\begin{tabular}[c]{@{}c@{}}Precio\\ (pesos)\end{tabular}} \\ \cline{4-5}
                                              &                             &                         & \begin{tabular}[c]{@{}c@{}}Tensión\\ (voltios)\end{tabular} & \begin{tabular}[c]{@{}c@{}}Corriente\\ (amperes)\end{tabular} &                                                                                  &                                                                                  &                                                                                            &                                                            &                                                                           \\ \hline
            \multirow{2}{*}{Bomba periférica} & \multirow{2}{*}{Fluvial}    & \multirow{2}{*}{Nero}   & \multirow{2}{*}{220 V}                                      & \multirow{2}{*}{2,4 A}                                        & \multicolumn{1}{c|}{\begin{tabular}[c]{@{}c@{}}Potencia\\ (vatios)\end{tabular}} & \multicolumn{1}{c|}{\begin{tabular}[c]{@{}c@{}}Altura\\ (metros)\end{tabular}}   & \multicolumn{1}{c|}{\begin{tabular}[c]{@{}c@{}}Caudal\\ (litros/\\ segundos)\end{tabular}} & \begin{tabular}[c]{@{}c@{}}Succión\\ (metros)\end{tabular} & \multirow{2}{*}{\$ 2.700}                                                 \\ \cline{6-9}
                                              &                             &                         &                                                             &                                                               & \multicolumn{1}{c|}{0,37 KW}                                                     & \multicolumn{1}{c|}{32 m}                                                        & \multicolumn{1}{c|}{0,55 l/s}                                                              & 9 m                                                        &                                                                           \\ \hline
            \multirow{2}{*}{Rele}             & \multirow{2}{*}{Songle}     & \multirow{2}{*}{SRD}    & \multirow{2}{*}{12 V}                                       & \multirow{2}{*}{30 mA}                                        & \multicolumn{2}{c|}{\begin{tabular}[c]{@{}c@{}}Tension\\ (voltios)\end{tabular}}                                                                                    & \multicolumn{2}{c|}{\begin{tabular}[c]{@{}c@{}}Corriente\\ (amperes)\end{tabular}}                                                                      & \multirow{2}{*}{\$ 100}                                                   \\ \cline{6-9}
                                              &                             &                         &                                                             &                                                               & \multicolumn{2}{c|}{240 V}                                                                                                                                          & \multicolumn{2}{c|}{10 A}                                                                                                                               &                                                                           \\ \hline
            \multirow{2}{*}{Transistor}       & \multirow{2}{*}{ON}         & \multirow{2}{*}{BC547}  & \multirow{2}{*}{0,6V}                                       & \multirow{2}{*}{0,15 mA}                                      & \multicolumn{1}{c|}{\begin{tabular}[c]{@{}c@{}}Potencia\\ (vatios)\end{tabular}} & \multicolumn{1}{c|}{\begin{tabular}[c]{@{}c@{}}Tension\\ (voltios)\end{tabular}} & \multicolumn{1}{c|}{\begin{tabular}[c]{@{}c@{}}Corriente\\ (amperes)\end{tabular}}         & \begin{tabular}[c]{@{}c@{}}Ganancia\\ ó hFE\end{tabular}   & \multirow{2}{*}{\$ 10}                                                    \\ \cline{6-9}
                                              &                             &                         &                                                             &                                                               & \multicolumn{1}{c|}{500 mW}                                                      & \multicolumn{1}{c|}{45 V}                                                        & \multicolumn{1}{c|}{100 mA}                                                                & 110                                                        &                                                                           \\ \hline
            \multirow{2}{*}{Regulador}        & \multirow{2}{*}{Fairchild}  & \multirow{2}{*}{LM7805} & \multirow{2}{*}{7-20V}                                      & \multirow{2}{*}{1 A}                                          & \multicolumn{2}{c|}{\begin{tabular}[c]{@{}c@{}}Tension\\ (voltios)\end{tabular}}                                                                                    & \multicolumn{2}{c|}{\begin{tabular}[c]{@{}c@{}}Corriente\\ (amperes)\end{tabular}}                                                                      & \multirow{2}{*}{\$ 50}                                                    \\ \cline{6-9}
                                              &                             &                         &                                                             &                                                               & \multicolumn{2}{c|}{5 V}                                                                                                                                            & \multicolumn{2}{c|}{1 A}                                                                                                                                &                                                                           \\ \hline
            \multirow{2}{*}{Microcontrolador} & \multirow{2}{*}{Microchips} & \multirow{2}{*}{12F675} & \multirow{2}{*}{5V}                                         & \multirow{2}{*}{0,15 mA}                                      & \multicolumn{1}{c|}{\begin{tabular}[c]{@{}c@{}}Reloj\\ (hertzios)\end{tabular}} & \multicolumn{1}{c|}{\begin{tabular}[c]{@{}c@{}}Pin rate\\ (amperes)\end{tabular}} & \multicolumn{1}{c|}{\begin{tabular}[c]{@{}c@{}}Pines E/S\end{tabular}}                     & \begin{tabular}[c]{@{}c@{}}Memoria\\(palabras)\end{tabular}& \multirow{2}{*}{\$ 200}                                                   \\ \cline{6-9}
                                              &                             &                         &                                                             &                                                               & \multicolumn{1}{c|}{4 MHz}                                                      & \multicolumn{1}{c|}{25 mA}                                                        & \multicolumn{1}{c|}{6}                                                                     & 1024                                                       &                                                                           \\ \hline
        \end{tabular}
        \caption{Componentes utilizados en el diseño de la aplicación}
        \label{tab:componentes}
    \end{table}
\end{landscape}}

\subsection{Enunciado}

A continuación transcribimos el enunciado original de la práctica del
que se extraen los requerimientos para el desarrollo.

\begin{quotation}

    \begin{center}
        \textit{\textbf{Taller de Microprocesadores: Trabajo 
        Practico Nro. 3}}
    \end{center}

    Diseñar un sistema que permita llenar y vaciar un tanque de líquido 
    con dos bombas independientes. 

    El inicio de llenado del tanque debe realizarlo mediante un botón 
    (o señal digital externa ) y deben ingresar 500 lts (o cierto volumen 
    fijo). El vaciado también puede realizarse con un botón o señal 
    digital externa. El llenado no puede realizarse si el tanque esta por 
    encima de cierto nivel, y el vaciado deber finalizar con un nivel 
    mínimo en el tanque para proteger que la bomba no funcione sin 
    líquido. El vaciado no puede en caso que no existe ese nivel mínimo y 
    debe indicarse con una luz que el tanque se encuentra en su mínimo 
    nivel.

\end{quotation}

\subsection{Plataforma de Desarrollo}

El microcontrolador es programado utilizando el lenguaje C. 
Presentamos como anexo una copia del código fuente.

El \textit{firmware} que se implanta en el microcontrolador es  
producido con el compilador MPLAB XC8\cite{bib:compilador} de la 
empresa Microchips, diseñado específicamente para la línea de 
microcontroladores de 8 bits a la que pertenece el PIC12F675.

El diseño y simulación del esquemático correspondiente se realiza 
con el software Proteus\cite{bib:simulador} de la compañía 
Labcenter Electronics.

\subsection{Listado de componentes}

En la Tabla \ref{tab:componentes} figuran los componentes 
utilizados en nuestro diseño, los cuales fueren elegidos tanto 
por sus prestaciones adecuadas a la aplicación como por su 
disponibilidad y costo en el mercado local.

Se dejan de lado los componentes pasivos como capacitores y 
resistores, los que de todas forman se indicaran en el esquemático
con sus respectivos valores.

\section{Diseño del sistema}

\subsection{Diagrama de arquitectura}

En la Figura \ref{fig:general} presentamos la topología 
general de nuestra aplicación. Se puede observa que 
la misma cubre una extensa geografía desde los sensores 
sumergidos en el tanque, pasando por la bomba de desagote
cercana al mismo, la bomba de carga más cercana a la toma,
y finalmente nuestra placa con las interfaces indicadoras 
y de accionado. 

\begin{figure}[h]\centering
    \includegraphics[height=12cm]{diagrama_sistema.jpg}
    \caption{Esquema general del sistema.}\label{fig:general}
\end{figure}
%\clearpage

También, del diseño se desprende que conviven señales de 
disparo como las correspondiente a las órdenes de carga y
descarga manuales, junto con señales de monitoreo continuo
como las del nivel del tanque.

Corresponde aquí que hagamos notar los desafíos que ya 
plantea este desarrollo: el multiplexado de las diferentes
señales de entrada correspondientes a niveles de agua, y 
la exigencia de trabajo de las señales que activan tanto el
llenado como el vaciado.

\subsection{Esquemático del circuito.}

El esquemático de nuestro circuito se presenta en la Figura
\ref{fig:esquematico} con las libertades en cuanto a las
posibilidades de simulación y prueba de los componentes 
activos.

En un principio vemos las indicaciones lumínicas y mandos 
con botones que ya desarrollamos en nuestros trabajos 
anteriores. También proveniente de nuestras prácticas
pasadas, contamos con una fuente de alimentación de 12 
voltios a la que agregamos un regulador a 5 voltios para el 
microcontrolador.

\begin{figure}[h]\centering
    \includegraphics[width=\textwidth]{tp3.jpg}
    \caption{Esquemático del circuito electrónico.}
    \label{fig:esquematico}
\end{figure}

La disponibilidad de ambas tensiones de alimentación es 
muy conveniente para abordar la solución al problema de 
la exigencia de trabajo, es decir, el circuito de salida 
que va desde el microcontrolador, pasando por los 
transistores que amplifican la corriente, y llega hasta 
los relé que finalmente sostienen la carga de corriente 
alterna directa de los tomas. 

Trabajar con el relé a 12 voltios nos permite reducir 
su demanda de corriente, si lo comparamos con los casi 
100 mili amperes que nos exigiría un relé de 5 voltios. 
Las prestaciones de un transistor BC547 son más que 
suficientes para amplificar, en saturación, una corriente
de menos de 300 micro amperes que es demandada al 
microcontrolador, cumpliendo así con las especificaciones 
del fabricante\cite{bib:datasheet}.

Finalmente, como se indicó en la Introducción, el método
más eficaz para la implementación de un conversor 
analógico digital que nos permita además multiplexar las
entradas al microcontrolador, es la escalera de 
resistencias del tipo ``R-R2'' que utilizamos con los 
sensores de nivel de agua.

\section{Análisis de funcionamiento.}

El firmware para nuestra aplicación, tiene como es 
habitual la forma de un ciclo infinito de espera para 
las señales disparadoras de los comandos de llenado y de 
vaciado del tanque (líneas de la 25 al final).

Ambas señales se testean doblemente en función de falsos
contactos (líneas 27 y 30 para el llenado, líneas 42 y 45 
para el vaciado). Una vez confirmado el comando, se activa
el circuito de fuerza (líneas 32 y 50, respectivamente) 
hasta que el monitoreo del módulo de conversión analógico 
digital que ya posee el microcontrolador, alcance los 
valores de referencia (líneas 34 y 53).

Como ha sido requerido para proteger la bomba, el comando
de vaciado no solamente se prohibe a partir de un nivel
límite, sino que además se informa la situación al usuario
con una señal luminosa (líneas 57 a 59).

\section{Conclusiones}

Nuestro trabajo ha abordado una cantidad de temáticas 
electrónicas, especialmente las referidas a la vinculación
de componentes analógicos con los digitales.

En muchos casos estas soluciones se realizaban a través 
de integrados con componentes activos como
diodos y transistores. Hoy, la baratura alcanzada por 
la reducción del costo de área de la electrónica digital
hace posible utilizar microcontroladores que agregan un
grado mayor de versatilidad a nuestros diseños.

Queremos hacer notar aquí también lo que ha sido nuestra 
experiencia en el cursado a distancia de una materia de
práctica de laboratorio. Si bien los simuladores no pueden
abarcar la cantidad de variables que intervienen en el 
funcionamiento de un circuito real, sentimos muy 
satisfactoria nuestra experiencia con el software 
presentado por la cátedra, lo mismo que el entrenamiento 
que la carrera nos ha dado en el uso de herramientas de 
trabajo en equipo como el uso de control de versiones, 
teleconferencias, etc.

\noindent\rule{\textwidth}{1pt}

\begin{thebibliography}{9}

\bibitem{bib:compilador}
\textit{``Microchip MPLAB XC8 C Compiler"}
(Versión 2.10; Microchip Technology Inc.: 2019).
Recuperado de https://www.microchip.com/mplab/compilers

\bibitem{bib:simulador}
\textit{``Proteus 8 Professional"} 
(Versión 8.5 Service Pack 0; Labcenter Electronics: 2016).
Recuperado de https://www.labcenter.com/

\bibitem{bib:datasheet}
Microchip (2010).
\textit{``PIC12F629/675 Data Sheet. 8-Pin FLASH-Based 8-Bit CMOS 
Microcontrollers''.}
EE.UU. Recuperado de 
http://ww1.microchip.com/downloads/en/DeviceDoc/41190G.pdf

\end{thebibliography}


\appendix

\section{}
A continuación listamos en extenso el código fuente en lenguaje 
C correspondiente al \textit{firmware} impreso en el 
microcontrolador. Su explicación se encuentra en la sección de 
Análisis de funcionamiento.

\lstinputlisting[language=C,basicstyle=\ttfamily\scriptsize,numbers=left]{../src/firmware.c}

\end{document}
