\documentclass{article}
% Eso define el tipo de documento.
% `article` se divide en secciones y tiene el resumen pegado al texto
\usepackage{graphicx}
% Son como los import de librerias. `graphicx` es para importar imagenes
\usepackage{listings}
% `listings` es para insertar los codigo fuente
\usepackage[utf8]{inputenc}
% `utf8` es para no tener problemas con el encoding (acentos, etc.)
\usepackage{mystyle}
% Nuestra configuracion

\title{Trabajo Práctico Nro. 2:\\“Puertos Analógicos”}
\author{Amodey, Leandro - leandroamodey@gmail.com
\and Monti, Matías - matiasmonti@hotmail.com
\and Quinteros, Fernando - lordfers@gmail.com
\and Araneda, Alejandro – eloscurodeefeso@hotmail.com}
\date{1er. Cuatrimestre 2020\\Lunes, 11 de Mayo, 8hs.}
% `title`, `author` y `date` es informacion que todos los archivos tienen.

\def\teacher{Ing. Jorge H. Doorn\\Ing. Matías Presso}
% Con `def` sirve para definir variables.

\begin{document}
% Arranca el documento
\begin{titlepage}
% Creamos un titulo a medida
\makeatletter 
% `makeatletter` es para usar `\@title`, `\@author` y `\@date`
\centering
% centrado
\includegraphics{logo.png}\par
% incluye el gráfico `par` señala fin de parrafo
{\Large Ingeniería en Computación \par}
% `Large` es el tamaño de letra, igual que `LARGE` y `huge`. 
% Uso llaves {} para que solo valga entre llaves.
\vspace{0.5cm}
% Espacio vertical
{\LARGE Laboratorio de Microprocesadores \par}
\vfill
% `vfill` es rellenado vertical para cubrir toda la pagina.
% Cuando hay varios en la página, se reparten el sobrante.
{\huge \@title \par}
\vfill
Grupo 2:\par
{\renewcommand{\and}{\par}\@author}
% `renewcommand` modifica los comandos.
% Acá cambio los `and` que hacen una tabla rara
% por marcas de final de parrafo `par`.
% Dejo los `and` en `author` porque es lo que se usa.     
\vfill
Práctica entregada:\par
\@date
\vfill
Docentes:\par
\teacher
\vspace{1cm}
\makeatother
\end{titlepage}

\begin{abstract}
En el presente trabajo ejemplificaremos mediante diversar aplicaciones 
la utilización de los módulos analógicos de un microcontrolador. A su 
vez, éstas ilustrarán la forma en que el diseño digital que en general 
pertenece al dominio de la corriente contínua y de baja tensión, se 
relaciona con los componentes electrónicos propios también de la 
corriente alterna y de mayor tensión.
\end{abstract}

\section*{Introducción}
Dummy text

\subsection*{Tecnologías CMOS y TTL}
% Fijarse que este tiene * para que no se numere.
Dummy text

\section*{Descripción de la Práctica}

\subsection*{Enunciado}
Dummy text

\subsection*{Plataforma de Desarrollo}
Dummy text. Ejemplo de fuente monospace {\ttfamily para un comando} en el medio del párrafo.
% `ttfamily` pone una fuente monospace. Sin las llaves cambiaria todo en adelante.

\subsection*{Instrucciones de Compilación y Ejecución}
Dummy text. Ejemplo de émfasis para una palabra en \textit{inglés} en el medio del párrafo.

\section*{Diseño y Simulación}
Dummy text. Como en la Figura \ref{imagen}

% --- Para insertar una imagen escriban de acá
\begin{figure}[h]
% La opcion `[h]` es para "here", pero puede cagarse en eso.
% No usar "la imagen que sigue" sino la referencia.
    \centering
    \includegraphics[height=1cm]{logo.png}
    \caption{Ejemplo de insertar imagen}
    \label{imagen}
\end{figure}
% --- hasta acá.

Para referenciar el código en un apéndice decir que está en \ref{codigo}

\section*{Conclusiones}
Dummy text que cita a la obra \cite{boylestad}

\noindent\rule{\textwidth}{1pt}
% Linea horizontal sin identacion y del ancho del texto

\begin{thebibliography}{9}
% Comienza la bibliografia
% El "9" indica que el espacio para imprimir "9" es de la mayor referencia

% Para agregar una entrada a la bibliografia repetir de aca
\bibitem{boylestad} 
Boylestad, R. \& Nashelsky, L. (2002). 
\textit{"Electronic devices and circuit theory"}.
Upper Saddle River, N.J: Prentice Hall.
% Hasta aca termina la referencia

\end{thebibliography}

\appendix

\section{}\label{codigo}
Ejemplo de como agregar código de otro archivo

\lstinputlisting[language=C,basicstyle=\ttfamily\scriptsize,numbers=left]{../ejercicio1.c}
% `lstinputlisting` inserta el codigo.
% Lo configuro para C, en monospace y tamaño chico, con numeracion a la izquierda.
% Recordar que el archivo tiene una direccion relativa a esta carpeta.
% Los de verdad van a estar en la carpeta padre "../ejercicio1.c"

\newpage
% Incluir salto de pagina de manera manual en cada apendice.
\section{}

\end{document}
