\documentclass{article}
\usepackage{graphicx}
\title{Trabajo Práctico Nro. 1:\\“GPIO en Microcontroladores”}
\author{Amodey, Leandro - leandroamodey@gmail.com
\and Monti, Matías - matiasmonti@hotmail.com
\and Quinteros, Fernando - lordfers@gmail.com
\and Araneda, Alejandro – eloscurodeefeso@hotmail.com}
\date{1er. Cuatrimestre 2020\\Lunes, 11 de Mayo, 8hs.}
\addtolength{\oddsidemargin}{-1cm}
\addtolength{\evensidemargin}{-1cm}
\addtolength{\textwidth}{2cm}
\addtolength{\topmargin}{-1cm}
\addtolength{\textheight}{2cm}
\begin{document}
\begin{titlepage}
\addtolength{\oddsidemargin}{-.5cm}
\makeatletter
\framebox{%
\begin{minipage}[c][.99\textheight][c]{\linewidth}
\centering
\includegraphics{logo.png}\par
{\Large Ingeniería en Computación \par}
\vspace{0.5cm}
{\LARGE Laboratorio de Microprocesadores \par}
\vfill
{\huge \@title \par}
\vfill
Grupo 2:\par
\begin{center}
\@author    
\end{center}
\vfill
Práctica entregada:\par
\@date
\vfill
Docentes:\par
Ing. Jorge H. Doorn\par
Ing. Matías Presso
\vspace{1cm}
\end{minipage}}
\makeatother
\end{titlepage}

\renewcommand{\abstractname}{Resumen}
\begin{abstract}
El siguiente informe tieno por objetivo siguiente informe tieno por objetivo siguiente informe tieno por objetivo siguiente informe tieno por objetivo siguiente informe tieno por objetivo siguiente informe tieno por objetivo siguiente informe tieno por objetivo siguiente informe tieno por objetivo.
\end{abstract}

%\renewcommand{\contentsname}{\centering Indice}
%\tableofcontents

\section{Introducción}

Dummy text

\subsection*{Otro Tema}

Dummy text

\section{Descripción de la Práctica}

\subsection{Enunciado}

Dummy text

\subsection{Plataforma de Desarrollo}

Dummy text

\subsection{Instrucción de Compilación y Ejecución}

Dummy text

\section{Diseño y Simulación}

Dummy text. Como en la Figura \ref{fig:eje}

\clearpage % Para que aperezca
\begin{figure}
    \centering
    \includegraphics[height=1cm]{logo.png}
    \caption{Ejemplo de insertar imagen}
    \label{fig:eje}
\end{figure}


Dummy text

\section{Conclusiones}

Dummy text

\section{Referencias}
Dummy text

\appendix
\renewcommand\thesection{Apéndice \Alph{section}}
\section{}

\section{}

\end{document}